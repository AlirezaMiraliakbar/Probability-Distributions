\documentclass{article}
\usepackage[a4paper, margin=1in]{geometry}
\usepackage{amsmath}
\usepackage{xcolor}
\usepackage[framemethod=TikZ]{mdframed}
\usepackage{graphicx}
\usepackage{multicol}

\mdfsetup{
  roundcorner=10pt,
  linewidth=1pt,
  linecolor=blue,
  backgroundcolor=gray!10,
}

\title{Probability Distributions Cheat Sheet}
\author{Alireza Miraliakbar}
\date{Spring 2024}

\begin{document}
\maketitle

\begin{multicols}{2}
[
\section*{Continuous Distributions}
]

% Normal Distribution
\begin{mdframed}
\textbf{Normal (Gaussian) Distribution}

\textit{Description}: Described by its mean ($\mu$) and standard deviation ($\sigma$), and has the classic "bell curve" shape.

\textit{PDF}: $f(x) = \frac{1}{\sigma\sqrt{2\pi}} e^{-\frac{1}{2}\left(\frac{x-\mu}{\sigma}\right)^2}$

\textit{Mean}: $\mu$

\textit{Variance}: $\sigma^2$

\begin{center}
  \includegraphics*[width=0.7\textwidth]{normal.png}
\end{center}

\end{mdframed}

% Exponential Distribution
\begin{mdframed}
\textbf{Exponential Distribution}

\textit{Description}: Models the time between events in a Poisson point process, i.e., events occur continuously and independently at a constant average rate.

\textit{PDF}: $f(x;\lambda) = \lambda e^{-\lambda x}$ for $x \geq 0$

\textit{Mean}: $\frac{1}{\lambda}$

\textit{Variance}: $\frac{1}{\lambda^2}$

\begin{center}
  \includegraphics*[width=0.7\textwidth]{exponential.png}
\end{center}

\end{mdframed}

\begin{mdframed}
  \textbf{Uniform Distribution}
  
  \textit{Description}: Models equally likely outcomes between two boundaries, $a$ and $b$. It is characterized by constant probability density over its support.
  
  \textit{PDF}: $f(x) = \begin{cases} 
    \frac{1}{b-a} & \text{for } a \leq x \leq b \\
    0 & \text{otherwise}
  \end{cases}$
  
  \textit{Mean}: $\mu = \frac{a + b}{2}$
  
  \textit{Variance}: $\sigma^2 = \frac{(b-a)^2}{12}$
  
  \begin{center}
    \includegraphics*[width=0.7\textwidth]{uniform.png}
  \end{center}
  
\end{mdframed}
\begin{mdframed}
    \textbf{Gamma Distribution}
    
    \textit{Description}: Models the time until an event occurs a specific number of times, given a continuous and constant arrival rate. It generalizes the exponential distribution.
    
    \textit{PDF}: $f(x; k, \theta) = \frac{x^{k-1}e^{-\frac{x}{\theta}}}{\theta^k \Gamma(k)}$ for $x > 0$
    
    \textit{Mean}: $\mu = k\theta$
    
    \textit{Variance}: $\sigma^2 = k\theta^2$
    
    \begin{center}
      \includegraphics*[width=0.7\textwidth]{gamma.png}
    \end{center}
    
\end{mdframed}
\newpage
\begin{mdframed}
    \textbf{Beta Distribution}
    
    \textit{Description}: Models the distribution of probabilities, representing all the possible values of probability when outcomes are distributed according to a binomial distribution. It is especially useful in Bayesian inference and is defined on the interval [0, 1].
    
    \textit{PDF}: $f(x; \alpha, \beta) = \frac{x^{\alpha-1}(1-x)^{\beta-1}}{B(\alpha, \beta)}$ for $0 \leq x \leq 1$
    
    \textit{Mean}: $\mu = \frac{\alpha}{\alpha + \beta}$
    
    \textit{Variance}: $\sigma^2 = \frac{\alpha\beta}{(\alpha + \beta)^2(\alpha + \beta + 1)}$
    
    \begin{center}
      \includegraphics*[width=0.7\textwidth]{beta.png}
    \end{center}
    
\end{mdframed}
\begin{mdframed}
    \textbf{Log-Normal Distribution}
    
    \textit{Description}: Models variables that are the product of many independent, identically distributed random variables. It is suitable for representing variables that cannot be negative and have a long right tail, such as income, length of time, and biological measurements.
    
    \textit{PDF}: $f(x; \mu, \sigma) = \frac{1}{x\sigma\sqrt{2\pi}} e^{-\frac{(\ln x-\mu)^2}{2\sigma^2}}$ for $x > 0$
    
    \textit{Mean}: $\mu = e^{\mu + \frac{\sigma^2}{2}}$
    
    \textit{Variance}: $\sigma^2 = (e^{\sigma^2} - 1)e^{2\mu + \sigma^2}$
    
    \begin{center}
      \includegraphics*[width=0.7\textwidth]{lognormal.png}
    \end{center}
    
\end{mdframed}
\begin{mdframed}
      \textbf{Chi-Squared Distribution}
      
      \textit{Description}: A special case of the Gamma distribution, used primarily in hypothesis testing and in the construction of confidence intervals. It describes the distribution of the sum of the squares of $k$ independent standard normal random variables.
      
      \textit{PDF}: $f(x; k) = \frac{1}{2^{\frac{k}{2}}\Gamma(\frac{k}{2})}x^{\frac{k}{2}-1}e^{-\frac{x}{2}}$ for $x > 0$
      
      \textit{Mean}: $\mu = k$
      
      \textit{Variance}: $\sigma^2 = 2k$
      
      \begin{center}
        \includegraphics*[width=0.7\textwidth]{chisquared.png}
      \end{center}
\end{mdframed}
\begin{mdframed}
\textbf{Weibull Distribution}
        
\textit{Description}: Commonly used in reliability engineering and failure analysis, the Weibull distribution is flexible, able to model various types of data. It is characterized by a scale parameter $\lambda$ and a shape parameter $k$, which dictate the distribution's spread and form, respectively.
        
\textit{PDF}: $f(x; k, \lambda) = \frac{k}{\lambda}\left(\frac{x}{\lambda}\right)^{k-1}e^{-(x/\lambda)^k}$ for $x > 0$
        
  \textit{Mean}: $\mu = \lambda \Gamma\left(1 + \frac{1}{k}\right)$
        
  \textit{Variance}: $\sigma^2 = \lambda^2 \left[\Gamma\left(1 + \frac{2}{k}\right) - \left(\Gamma\left(1 + \frac{1}{k}\right)\right)^2\right]$
        
  \begin{center}
    \includegraphics*[width=0.7\textwidth]{weibull.png}
  \end{center}
\end{mdframed}

              
\end{multicols}
\begin{mdframed}
  \textbf{F-Distribution}
  
  \textit{Description}: Used primarily in the analysis of variance (ANOVA), comparing two variances, and in regression analysis. The F-distribution arises from the division of two chi-squared distributions divided by their respective degrees of freedom.
  \begin{multicols}{2}
    \textit{PDF}: $f(x; d_1, d_2) = \frac{\sqrt{\frac{(d_1 x)^{d_1} \cdot d_2^{d_2}}{(d_1 x + d_2)^{d_1 + d_2}}}}{x \cdot B\left(\frac{d_1}{2}, \frac{d_2}{2}\right)}$ for $x > 0$
  
  \textit{Mean}: $\mu = \frac{d_2}{d_2 - 2}$ for $d_2 > 2$
  
  \textit{Variance}: $\sigma^2 = \frac{2d_2^2(d_1 + d_2 - 2)}{d_1(d_2 - 2)^2(d_2 - 4)}$ for $d_2 > 4$
  \begin{center}
    \includegraphics*[width=0.3\textwidth]{fdistribution.png}
  \end{center}
          
  \end{multicols}
  
  
\end{mdframed}
\begin{multicols}{2}
[
\section*{Discrete Distributions}
]

% Bernoulli Distribution
\begin{mdframed}
\textbf{Bernoulli Distribution}

\textit{Description}: Models a single trial with two possible outcomes (success with probability $p$ and failure with probability $1-p$).

\textit{PMF}: $P(X=k) = p^k(1-p)^{1-k}$ for $k \in \{0,1\}$

\textit{Mean}: $\mu = p$

\textit{Variance}: $\sigma^2 = p(1-p)$

\begin{center}
  \includegraphics*[width=0.7\textwidth]{bernoulli.png}
\end{center}

\end{mdframed}

% Binomial Distribution
\begin{mdframed}
\textbf{Binomial Distribution}

\textit{Description}: Models the number of successes in a fixed number of independent Bernoulli trials, each with the same probability of success.

\textit{PMF}: $P(X=k) = \binom{n}{k}p^k(1-p)^{n-k}$ for $k = 0, 1, ..., n$

\textit{Mean}: $\mu = np$

\textit{Variance}: $\sigma^2 = np(1-p)$

\textit{Figure Placeholder}

\end{mdframed}

\end{multicols}

\end{document}
